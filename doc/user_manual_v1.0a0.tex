\documentclass[a4paper, 10pt]{article}

\begin{document}

\bibliographystyle{unsrt}

\title{\LARGE SpecSwap-RMC Version 1.0a0 \\ User Manual}

\date{2013-01-21}

\author{Mikael Leetmaa \hspace{5mm} $<${\tt leetmaa@fysik.su.se}$>$}

\maketitle

\tableofcontents

\section{About this document}
This document is intended to serve as a guideline for the use and input of the
SpecSwap-RMC program. It is not intended as a tutorial or a
full fledged description of the SpecSwap-RMC method. But it is still
our hope that it can serve as a useful guidance to get started with
setting up your first SpecSwap-RMC simulations. And if ever in doubt
about the intended behavior of a particular input keyword this
document should be consulted.
The aim has been to write an as up to date and
accurate description as possible of all the input parameters and file
formats, however there might still be errors
in this document. If you believe a keyword or format behaves
differently from what is described here you could check the source
code to see what it really does. And if you do find bugs, issues,
inconsistencies or errors in this
document, or in the source code, please don't hesitate to contact the SpecSwap-RMC authors so that
the problem can be fixed. You find our contact information in the
README file in the source code distribution.

\section{The method in brief}
The SpecSwap-RMC method \cite{SS-RMC} is a flavor of reverse Monte-Carlo (RMC)
\cite{RMC} that makes use of a finite discrete set of local
structures, for which all relevant data is pre-computed, to
expand the configuration space in these structures. The main reason for using the SpecSwap-RMC method
compared to conventional RMC is to avoid expensive re-calculation of
the data, making it practically manageable to fit data sets for which the
theoretical evaluation is very CPU expensive. SpecSwap-RMC thus makes it
possible to use RMC with data such as full multiple scattering EXAFS,
XAS and NMR, where the nature of the problem and the need for accuracy
demands expensive electronic-structure calculations, typically too costly to
include in a straight forward way in a conventional RMC simulation.
\\\\
All relevant data for the local structures are
pre-computed. A sub-set of the available local configurations (the
{\it sample set}) is taken
out to represent the configuration of the system. A trial move in the
Monte-Carlo run is performed by replacing ( ``swapping'') a local
configuration present in the sub-set with a local structure that is
not. The re-calculation of the data for the new configuration of the
system (needed for comparing against a provided reference to evaluate
the success of the move - by far the most time consuming step of a
conventional RMC simulation) is reduced to a simple add and
subtract operation of the data that
corresponds to the swapped local structures.
\\\\
When equilibrium is reached in the SpecSwap-RMC simulation (this is
typically achieved very fast given the parameters are set to yield a
reasonable acceptance rate) the {\it sample set} is probed repeatedly on a given
interval to record which local structures are present. Divided by the
number of times the {\it sample set} was probed these records
converge towards a weighting of each local structure proportional to its
importance within the total set of local structures to represent/fit the
reference data. These weights are the main result from the
SpecSwap-RMC run. Structural properties such as pair-correlation
functions and angle distributions can then be calculated and compared
weighted and unweighted
from the total set of local structures. For further details about the
method see ref \cite{SS-RMC} and for an example of the method in use see ref \cite{SS-RMC-EXAFS}.

\section{Naming conventions}
This document (and also the comments in the source code) will try to
be as consistent as possible with the naming
conventions of ref \cite{SS-RMC}.

\subsection{library}
We will use the term {\it library} for
all available local structures taken together, each with its
corresponding pre-calculated data and compiled in to a binary
file. The {\it raw library} or {\it library raw data} are the files
making up the library before being compiled into a binary library
file.

\subsection{basis set}
The set of all available local structures in the library
 will be referred to as the {\it basis set}.

\subsection{basis element}
Each local structure with
 its corresponding pre-calculated data will be called a {\it basis
   element}.

\subsection{sample set}
The sub-set of the {\it basis set} that, at any
 given moment during the SpecSwap-RMC run, represents the system
 configuration, is referred to as the {\it sample set}.

\section{Before you start}
Since the SpecSwap-RMC method uses pre-computed data from a (large)
set of local structures you will need to select the local structures
and generate data before you start the simulation.

\subsection{Finding suitable basis structures}
For a successful (and meaningful) SpecSwap-RMC simulation the
structures taken as the basis set must be chosen with some care.
They must be close enough to the real structure (the structure
corresponding to the reference data modeled) to get a reasonable fit
without a too low acceptance rate, yet be diverse enough
for having the possibility of representing the real diversity of the
local geometries in the material under study.

All to date practical applications of the SpecSwap-RMC method known to
the authors has used structures cut out from MD simulations, conventional RMC
simulations, or generated via random distortions around a known or
assumed equilibrium geometry see e.g.\ refs \cite{SS-RMC, SS-RMC-EXAFS}.
However, at the time of
writing there exists no standard way to obtain
suitable basis structures, so the best advice is to be careful and use your
creativity.
\\\\
The single most important aspect when constructing a {\it basis set} library
is that it must be possible to fit the data. If no satisfactory fit
can be obtained the basis set must be expanded. Consider a case with a {\it basis set} containing 10000 {\it basis elements}
and a {\it sample set} of 100 {\it basis elements}, and we assume for
the argument that the
data has been calculated with infinite accuracy. If it turns out
impossible to obtain a reasonable fit to the reference that would  be equivalent
to say that there are no (or way too few) combinations of 100 {\it
  basis elements} drawn from the set of 1000 that can represent the
data. This means there are not enough {\it basis elements} with a
geometry corresponding to the reference or that we are missing some
very important structural components in the {\it basis set}.
If the fit improves drastically when reducing the size of the {\it
  sample set} this is typically an indication of an insufficient
basis set. There are simply to few {\it basis elements} with some
important quality missing.
The weights can is in such a case still be quite useful.
Examining the structures of the highest weighted {\it basis elements}
would give an indication of what type of structures are missing in the
{\it basis set}.

\subsection{Computing the data}
For each chosen local geometry all relevant data must be calculated
before the library is generated,
with one important exception. When generating the library (see below),
the {\tt mklib} program calculates and stores the distributions of
distances with respect to the atoms defined as the central molecule
for each basis element.

Since all other data to use in SpecSwap-RMC must be computed before
the library is generated the
program is necessarily general such that it can not know what the
data represents. Any type of data that can be stored as a function of
one variable, i.e.\ any type of curve data, can be included. Any higher
dimensional data must be mapped down to two dimensions before
inclusion is possible. Two dimensional data goes here under the name
{\it curve data}.
Data represented as a single number on each {\it basis element} is
named {\it scalar data}. {\it Scalar data} could e.g.\ be a frequency, a number
classifying the local structure, the local density or Voronoi
polyhedra volume.
All data included in the fit (except the pair correlation
functions) must be represented either as {\it curve
  data } or {\it scalar data}.

 \subsubsection{A note on accuracy}
It is important to remember that the accuracy of the results from
structure modeling with RMC is
determined by the accuracy with witch the data to fit can be
calculated.
Structure modeling with SpecSwap-RMC is not and exception. If
the data is not calculated accurately enough from a given structure,
the resulting structure from the RMC simulation will not correspond to
the real physical structure, and conclusions drawn
can be quite wrong. For a sound use of RMC one needs to be aware
of the particular limitations and error bars for the type of data
which is used for the fit, and conclusions drawn should always be
evaluated with the accuracy of the method used to
calculate the data in mind. For SpecSwap-RMC simulations it is also
crucial that a good enough {\it basis set} library is used.

\section{Generating a library with Mklib}
When all data is calculated form the chosen local structures to use as
{\it basis set} the data needs to be compiled into a library file together
with the corresponding geometries, to be used in the SpecSwap-RMC
simulation.  The SpecSwap-RMC program uses a binary file format
defined by the binary I/O
implementation in the {\tt library.cpp} file. The SpecSwap-RMC
software package includes a program,
{\tt mklib}, that can generate the binary library files from the raw
library data.
\\\\
To generate the library several data files are needed for each {\it basis element}; a file holding the geometry, a file holding the curve
data for each curve and a file holding all scalar data. The files
corresponding to the same {\it basis element} must have the same base
name, with endings corresponding to the names of the curves, {\tt .xyz} for
the geometry and {\tt .sclr} for the scalars. For each curve
data a file giving the scale is also needed. All files are then placed
in a folder with the same name as the intended library name, and a
manifest file, {\tt libraryname.info} (the {\it info file}) with
specific info about the library and the constituent basis elements is placed
in the directory. The syntax of the {\tt .info} file is rather strict
fixed format so extra care must be taken to type every thing in
correctly. The {\tt mklib} program can then, with the information from
the {\tt info} file, be used to compile all data into a binary library file
for use in SpecSwap-RMC, by typing:\\\\
{\tt ./mklib3.0.x  -c libraryname }\\\\
This makes {\tt mklib} read the file {\tt libraryname.info}
from the {\tt ./libraryname} folder where also all other files to
compile must be present. Please see \ref{mklib-input} below, as well as
the examples shipped in the {\tt
  /functest/testmklib} folder in the SpecSwap-RMC source, for details.

\section{Mklib input}
\label{mklib-input}
This describes the input format of the {\tt .info} file for {\tt mklib},
as well as the files refered to from that file.

\subsection{Curves}
The first line in the {\tt .info} file is the\\\\
{\tt NCURVES N}\\\\
keyword, where N is the number of specific curve data types to include
in the library, e.g.\ 2 if including both say EXAFS and XAS.\\\\
After the {\tt NCURVES N} line a set of four lines appear N times,
once for each curve. These are the lines:\\\\
{\tt CONVOLUTE D1 D2 D3 D4}\\
{\tt FORMAT start stop steps}\\
{\tt SCALE scale\_filename}\\
{\tt ENDING ending}\\\\
where {\tt D1 D2 D3 D4} is any set of four floating point numbers for
the user to classify the curve. The name of the keyword stems from the
use of a particular type of broadening scheme for TP-XAS
calculations. The values are stored in the library file but in the
present version they are not used. The {\tt start}, {\tt stop}
ands {\tt steps} values for the {\tt FORMAT} keyword specifies the
first and last x values and the number of data points to expect in
each curve file (one for each {\tt basis element}, see below) of the type given as
parameter to the {\tt ENDING} keyword. If say EXAFS is fitted the
ending could e.g.\ be {\tt .exafs}, and the keyword would the be given as\\\\
{\tt ENDING exafs}\\\\
All files with the {\tt .exafs} extension would then need to have the
data format with two columns, the first column giving the x values,
with the first and last values as well as number of lines as specified
with the {\tt FORMAT} keyword.
\\\\
The argument to the {\tt SCALE} keyword is the file name of a data
file to use as scale for the x axis. The scale file must have the
number of data lines (the same value as the {\tt steps} value to the
{\tt FORMAT} keyword) on the first line. The data must be given in the
correct number of lines, in two columns. The x values (the data in the first column)
will be stored in the library while the values in the second column
will not.
All basis element curve data of the
same type must match this x scale, and all reference curves of this
type used in subsequent SpecSwap-RMC runs with this library must also
match in x values. This is to ensure that no mistakes are made by
comparing data which are not normalized to the same x scale.
\\\\
As an example, if we fit EXAFS and XAS the whole curves part of the
{\tt mklib} input file could look like this:
\\\\
{\tt NCURVES 2}\\
{\tt CONVOLUTE 1.0 2.0 1.0 3.0}\\
{\tt FORMAT 2.3 7.65 108}\\
{\tt SCALE dummy\_exafs\_ref.data}\\
{\tt ENDING exafs}\\
{\tt CONVOLUTE 0.5 535.0 1.0 545.0}\\
{\tt FORMAT 530.0 550.0 201}\\
{\tt SCALE xas\_scale\_file}\\
{\tt ENDING xas}\\\\
If no curves data should be included in the library the line {\tt
  NCURVES 0} is given at the top, followed directly by the scalars information
(see below).

\subsection{Scalars}
After all the curve information is given the scalar information is
presented in two lines as:\\\\
{\tt NSCALARS N}\\
{\tt NAMES name1 name2 ... nameN}\\\\
Where {\tt N} is the number of scalars and a unique name for each
scalar must be given as a space separated list after the {\tt NAMES}
keyword. Say that we have two scalars, the Voronoi polyhedra volume
and area this could look like:
\\\\
{\tt NSCALARS 2}\\
{\tt NAMES VP-area VP-volume}\\\\
If there are no scalar data in the library {\tt NSCALARS 0} should be
given and the {\tt NAMES} line should be skipped.

\subsection{Geometry}
The scalars input is followed by a line indicating how many atom types
there are. The names of the elements could be any strings but it
important that the atom type names correspond to the data given later
in the geometry {\tt .xyz} files. The format is:
\\\\
{\tt NATOMTYPES N name1 name2 ... nameN}
\\\\
If all local configurations have the same number of atoms and if the same
indices should be considered included in the central molecule an
optional keyword {\tt ATOMS} should be given after the {\tt
  NATOMTYPES}. The {\tt ATOMS} keyword takes the arguments:\\\\
{\tt ATOMS M M1 M2 ... MN MOL Q Q1 Q2 ...}\\\\
where {\tt M} is the number of atoms (the same for all {\tt .xyz} files),
{\tt M1} the number of atoms of the first atom type, assumed to be the first
{\tt M1} entries in the {\tt .xyz} file, {\tt M2} the number of atoms
of the second atom type, etc. After the last integer indicating the
number of atoms of a specific type the word {\tt MOL} is written,
followed by the number of atoms in the central molecule {\tt Q}, and
then {\tt Q} integers telling the indices (atom numbers indexed from zero)
in the {\tt .xyz} of
the central molecule atoms. Note that the first index in this list should
always be 0 (zero). To clarify with an example, if we have 39 water
molecules the input could look like this.\\\\
{\tt NATOMTYPES 2 O H}\\
{\tt ATOMS 117 39 78 MOL 3 0 39 40}

\subsection{Basis name listing}
Lastly the keyword {\tt START} is given followed by the number of
basis elements to compile into a library. Then all the base names
(without endings) are given, one on each line, followed by the final
line with the word {\tt END}. In a realistic situation this list is
likely to be several thousand lines long but we give here as an example of the
format a basis name listing with 5 entries.\\\\
{\tt START 5}\\
{\tt example\_basename1}\\
{\tt example\_basename2}\\
{\tt example\_basename3}\\
{\tt example\_basename4}\\
{\tt example\_basename5}\\
{\tt END}

\subsection{Data for each basis element}

The data for the library is given in files named after the listed
base names of the basis elements, with endings corresponding to the
curve, scalar and xyz data.

\subsubsection{Curve data}
For each curve and listed basis name there must be a file name {\tt
  basename.ending} giving the curve data. In our above example with
{\tt .xas} and {\tt .exafs} data endings the files for the first
listed base name would be:\\\\
{\tt example\_basename1.xas}\\
{\tt example\_basename1.exafs}\\\\
The data in these files are given in two columns with the number of
lines corresponding to the {\tt steps} value given to the {\tt FORMAT}
keyword in the {\tt .info} file. The first column data must match the
given values in the corresponding scale file.

\subsubsection{Scalar data}
For each basis element there must be a file named
{\tt basename.sclr} holding scalar information. The scalars are given
as a space separated list of floating point numbers on one line, one
number for each scalar in the system. All basis elements must have the
same number of scalar values, as specified in the {\tt .info} file.

\subsubsection{Geometry data}
The geometry data from each basis element is
given in a file {\tt basename.xyz}. The format of this file is the
``standard'' xyz format, the first line gives the number of
atoms. Followed by a blank line. Then one line for each atom with the
symbol and x, y and z coordinates.\\\\
{\tt N}\\\\
{\tt symbol1 x1 y1 z1}\\
{\tt symbol1 x2 y2 z2}\\
{\tt ... }\\
{\tt symbolN xN yN zN}\\\\
The types and number of atoms must match the {\tt NATOMTYPES} and {\tt ATOMS}
information given in the {\tt .info} file.

\subsection{A full example}
Putting all the above examples together the content of the {\tt .info}
file looks like this:\\\\
{\tt NCURVES 2}\\
{\tt CONVOLUTE 1.0 2.0 1.0 3.0}\\
{\tt FORMAT 2.3 7.65 108}\\
{\tt SCALE dummy\_exafs\_ref.data}\\
{\tt ENDING exafs}\\
{\tt CONVOLUTE 0.5 535.0 1.0 545.0}\\
{\tt FORMAT 530.0 550.0 201}\\
{\tt SCALE xas\_scale\_file}\\
{\tt ENDING xas}\\
{\tt NSCALARS 2}\\
{\tt NAMES VP-area VP-volume}\\
{\tt NATOMTYPES 2 O H}\\
{\tt ATOMS 117 39 78 MOL 3 0 39 40}\\
{\tt START 5}\\
{\tt example\_basename1}\\
{\tt example\_basename2}\\
{\tt example\_basename3}\\
{\tt example\_basename4}\\
{\tt example\_basename5}\\
{\tt END}\\\\
And for this particular example it is also required that the files {\tt
  dummy\_exafs\_ref.data} and {\tt xas\_scale\_file} are present, as well as for each base name the files with endings: {\tt
.exafs .xas .sclr .xyz}

\section{Running a simulation}
Once all data is calculated and the library is set up running the
actual SpecSwap-RMC simulation is fairly straight forward. The
SpecSwap-RMC executable is run with the name of the input file {\tt name.inp}
withouth the {\tt .inp} ending as the only argument, and the output to
stdout can be sent to an output file of choice:\\\\
{\tt ./SpecSwapRMC.x name $>$ output }\\\\
See \ref{input} below for a description of the input file
format. For a few proper working examples please see the examples
in the {\tt  functest/testspecswap } folder in the SpecSwap-RMC source
distribution.

\section{Analyzing results}
When the SpecSwap-RMC simulation is finished i.e.\ when it has run for a
long enough time for the weights to approach convergence, there is a
single, yet very useful, built in analysis option.
The typical usage is to make a new
restart from the converged weights, but instead of running the
Monte-Carlo loop use the option available in the ANALYSIS input
section to generate weighted and unweighted curves. See
\ref{analysis-input} below for more details.
See also the examples in the {\tt  functest/testspecswap
} folder in the SpecSwap-RMC source distribution.


\section{SpecSwap-RMC input}
\label{input}
The SpecSwap-RMC program takes the name of the input file without the
{\tt .inp} extension as the only command line argument. This file
specify fully the behavior of the program.

\subsection{Input structure}
The input file is structured in keyword sections, each with a specified set of
possible keywords.

\subsubsection{Sections}
Each section begins with a \&
character followed by its name, and ends with the \&END
keyword. Inside the section are one or more keywords given, some of
which are optional, as described in more detail below. Some sections
also take a keyword argument directly after the name.\\\\
\&SECTIONNAME $<$argument$>$\\
KEYWORD1  $<$arg1 arg2 ...$>$\\
KEYWORD2  $<$arg1 arg2 ...$>$\\
...\\
\&END\\
\\\\
The first section is always the RUN section which controls the general
flow of the program. For each curve fit there must be a CURVE
section and for each scalar fit there must be a SCALAR section.
If fitting
a pcf there is a PCF section to add and if interested in post run
analysis one adds an ANALYSIS section.
 Comments starting with a \# character are allowed in the
input file between but not inside the sections.

\subsubsection{Keywords and arguments}
Each section has a particular set of valid keywords. Most keywords
take one more arguments as specified below. Also some of the
sections take arguments, e.g. the SCALAR section, to determine which
scalar to fit. Below is a list of all sections and their keywords.

\subsection{\&RUN}
The RUN section control all basic run parameters, how many steps to
take, which library file to use, when to save results, etc. A RUN
section  must be present as the first section in any input file.
Below is a listing of all valid RUN section keywords.

\subsubsection{TITLE $<title>$}
Set a title to the run. This title is used as the base name of the
generated restart and weights files. This keyword is compulsory. The
title must be given as a single string (no spaces).

\subsubsection{LIBPATH $<filepath>$}
Set a path to a library file to use for the run. This
keyword is compulsory. The path is given
relative to the directory in which the program is executed.

\subsubsection{SEED $<value>$}
Set a seed value to the random number generator. This keyword is
compulsory. The value must be given as a positive integer.

\subsubsection{NSAMPLE $<value>$}
Set the number of elements in the sample set. This keyword is
compulsory. The value must be given as a positive integer.

\subsubsection{MOVES $<value>$}
Set the limit of the run, given as the number of {\it attempted} moves to take.
This keyword is compulsory. The value must be given as a positive integer.

\subsubsection{PRINT $<value>$}
Set the interval, in number of {\it accepted} moves, for printing to screen / stdout.
This keyword is compulsory. The value must be given as a positive integer.

\subsubsection{PROBE $<value>$}
Set the interval, in number of {\it accepted} moves, for probing the sample
set to collect weights. This keyword is compulsory. The value must be given as a positive integer.

\subsubsection{DUMP $<value>$}
Set the interval, in number of {\it accepted} moves, for writing the
momentaneous curve and pcf averages
from the sample set. This keyword is compulsory.
 The value must be given as a positive integer.

\subsubsection{RESTART $<file path>$}
Set the path to a restart file to make a restart from a previous
run. This keyword is optional. With no RESTART keyword given the run
will start from a random samples set, with all weights starting from
zero. The file path is given relative to the folder the program is
executed from.

\subsection{\&PCF}
The PCF section is added to the input file to fit a reference
pair correlation function. These are the valid keywords:

\subsubsection{RMIN $<value>$}
Give the first r value of the reference. This keyword is compulsory.

\subsubsection{RMAX $<value>$}
Give the last r value of the reference. This keyword is compulsory.

\subsubsection{DR $<value>$}
Give the spacing between the bins. This keyword is compulsory. It is
important that the number of bins as calculated from (RMAX-RMIN)/DR
is integer.

\subsubsection{FIT $<lower value>$ $<upper value>$}
Give the r interval to fit. This keyword is optional. If not given the whole
interval from RMIN to RMAX will be used.

\subsubsection{SIGMA $<value>$}
The sigma value to use in the fitting. This keyword is compulsory. The
value must be given as a positive floating point number.

\subsubsection{NUMBERDENSITY $<value>$}
The number density needed for normalizing the PCFs. This keyword is compulsory. The
value must be given as a positive floating point number.

\subsubsection{PARTIAL $<first type> <second type>$}
Specify the atom types of the partial to fit. This keyword is compulsory.
The atom types are referred
to in the order they have in the library, i.e., as given in the {\tt
  .info} file when the library was setup.

\subsubsection{PATH $<file path>$}
The path to the file holding the reference data. This
keyword is compulsory. The path is given
relative to the directory in which the program is executed.
The reference data file should have two columns, the first giving the
r points and the second the corresponding g(r) data. The bin size and
max and min values must match the values given in the RMIN and RMAX keywords.

\subsection{\&CURVE $<name>$}
The CURVE section is added to the input file to fit a reference
curve. The curve name, as present in the library (given as the ending
of this curve type in the {\tt .info} file when constructing the
library) is given as argument. Valid keywords are:

\subsubsection{EXPPATH $<file path>$}
The path to the file holding the reference data. This
keyword is compulsory. The path is given
relative to the directory in which the program is executed.
The reference data file should have two columns, the first giving the
x points and the second the corresponding curve points.
The bin size and max and min values must
match the values given in the {\tt .info} file when creating the library.

\subsubsection{AREARENORM}
This optional keyword works as a switch. If the keyword is present
this indicates that the
fitted curve should be re-normalized to the same area
(computed as the sum of all y values multiplied by bin size,
thus assuming all values to be positive)
before compared to the reference.

\subsubsection{SIGMA $<value>$}
The sigma value to use in the fitting. This keyword is compulsory. The
value must be given as a positive floating point number.

\subsection{\&SCALAR $<name>$}
The SCALAR section is added to the input file to fit a reference
scalar, either as a mean value, a specific value or as a distribution.
The scalar name, as present in the library (given as the name
of the scalar type in the {\tt .info} file when constructing the
library) is given as argument.
\\\\
The SCALAR section comes in three different flavors, depending on what
type of fit is to be performed. This is controlled by adding one of
the three keywords MEAN, VALUE or DISTRIBUTION as the first keywords in
the SCALAR section. Only one of these keywords can be added to a
given SCALAR section, but there is no limit to how many different SCALAR
sections there can be in one input file.

\subsubsection{MEAN $<target>$}
To fit the mean value of the specified scalar. The target value is
given as a floating point number.

\subsubsection{VALUE $<lower limit>$ $<upper limit>$ $<target>$}
To fit to a fraction within a certain interval. The target value refers
to the fraction of all element in the sample set that should have their
scalar value between the given lower and upper limit.

\subsubsection{DISTRIBUTION $<file path>$}
To fit a reference distribution. The file path is given
relative to the directory in which the program is executed.
The reference data file should start with an integer on the first
line giving the number of subsequent lines of data. The data should
have three columns, the first giving x (bin) values and the second the
distribution value. The third column gives a floating point number
which will be multiplied with the error in that point/bin to allow for
weighted error values. This is particularly useful if one wants to fit
harder to some parts of the reference distribution.
The distribution is area normalized
with the corresponding scalar distribution when comparing. The first
and last bins includes every thing below and above the
distribution and the x (bin) values in the first column refer to the
center of the bin.

\subsubsection{SIGMA $<value>$}
The scalar section must have the SIGMA keyword with the specified
value. This keyword is compulsory.

\subsection{\&ANALYSIS}
\label{analysis-input}
The ANALYSIS section is added to the input file to perform an analysis
run after the Monte-Carlo run has finished. The ANALYSIS section only takes one keyword.

\subsubsection{CHUNKS $<number of chunks>$ $<chunk size>$}
This keyword is
compulsory. All {\it basis elements} in the library are sorted according to
their weight. The {\it basis elements}
are then bunched into chunks and for all data sets, (curves,
scalars, pcfs) specified in the input file the data for each chunk is
written weighted and unweighted to file. The number of chunks and
chunk size are given as arguments. Note that if the number of chunks times the
chunk size is larger than the number of elements in the basis set
library the last chunk will be truncated accordingly.

\section{Additional sources of information}
For a source code reference manual see the document
{\tt refman\_vX.X.pdf} distributed with the SpecSwap-RMC source code.
For a hint on how to extend the program to fit other types of data,
or how to extend it with custom analysis functions there might be a
development manual written in the future. Contact the SpecSwap-RMC
authors if you are interested in such a document or further guidance.

\newpage
\bibliography{ref}

\end{document}

